\documentclass[hyperref=unicode]{beamer}

\usepackage[absolute,overlay]{textpos}
\usepackage{graphicx}
\usepackage{adjustbox}
\usepackage{mhchem}
\usepackage{wrapfig}
\usepackage{caption}
\usepackage[utf8]{inputenc}

\usepackage{animate}

\setbeamerfont{footnote}{size=\tiny}

\adjustboxset*{center}

\mode<presentation>{\usetheme{default}}
\DefineNamedColor{named}{pozadi}{RGB}{200,200,200}
\usecolortheme{crane}

\setbeamertemplate{footline}[frame number]
\usepackage{fontspec} 
\usepackage{unicode-math}

\usepackage{polyglossia}
\setdefaultlanguage{czech}

\def\uv#1{„#1“}

\addtobeamertemplate{frametitle}{
   \let\insertframetitle\insertsectionhead}{}
\addtobeamertemplate{frametitle}{
   \let\insertframesubtitle\insertsubsectionhead}{}

\makeatletter
  \CheckCommand*\beamer@checkframetitle{\@ifnextchar\bgroup\beamer@inlineframetitle{}}
  \renewcommand*\beamer@checkframetitle{\global\let\beamer@frametitle\relax\@ifnextchar\bgroup\beamer@inlineframetitle{}}
\makeatother
\setbeamercolor{section in toc}{fg=blue}
\setbeamertemplate{section in toc shaded}[default][100]

\title[Crisis]
{Porozimetrie}

\author
{Zdeněk Moravec}
\date[KPT 2004]
{hugo@chemi.muni.cz}

\begin{document}

\frame{\titlepage}

\section{Porozimetrie}
\frame{
	\frametitle{}
	\vfill
	\begin{columns}
	\begin{column}{.6\textwidth}
		\begin{itemize}
			\item Skupina metod pro stanovení měrného povrchu a porosity materiálů.
			\item \textbf{Měrný povrch} - povrch materiálu vztažený na jednotku hmotnosti.
			\item \textbf{Rtuťová intruzní porozimetrie} -- založena na vtlačování rtuti do pórů. Používá se pro stanovení pórů o velikostech od 4~nm do stovek mikrometrů. 
			\item \textbf{Plynová porozimetrie} -- sorpce plynu na povrch vzorku. Používá se pro stanovení pórů o velikostech od 0,33~nm do stovek nanometrů.
		\end{itemize}
	\end{column}
	\begin{column}{.5\textwidth}
		\begin{figure}
			\adjincludegraphics[width=.9\textwidth]{img/ZIF-8.png}
			\caption*{Struktura porézního zeolitu ZIF-8.\footnote[frame]{Zdroj: \href{https://commons.wikimedia.org/wiki/File:Structure_of_ZIF-8(H).png}{François-Xavier Coudert/Commons}}}
		\end{figure}
	\end{column}
	\end{columns}
	\vfill
}

\frame{
	\frametitle{}
	\vfill
	Měrný povrch je důležitou charakteristikou katalyzátorů, sorpčních materiálů, apod.
	\begin{columns}
	\begin{column}{.5\textwidth}
	\begin{tabular}{|l|l|}
	\hline
	\textbf{Materiál} & \textbf{SA [m$^2$.g$^{-1}$]} \\\hline
	MOF & 7140 \\\hline
	Grafen & 2700 \\\hline
	Aktivní uhlí & 500--3000 \\\hline
	MCM-41 (\ce{SiO2}) & 1000 \\\hline
	Molekulová síta & až 1000 \\\hline
	Faujesite & 900 \\\hline
	Alumina & 200 \\\hline
	\ce{CaCO3} & 3 \\\hline
\end{tabular}

	\end{column}
	\begin{column}{.5\textwidth}
		\begin{center}
			\begin{figure}
				\adjincludegraphics[width=.8\textwidth]{img/Radial_mesoporous_silica.jpg}
				\caption*{Mezoporézní silica.\footnote[frame]{Zdroj: \href{https://commons.wikimedia.org/wiki/File:Radial_mesoporous_silica.jpg}{Xin Min et al/Commons}}}
			\end{figure}
			
			\textbf{Velikost pórů:}
			
			\begin{tabular}{ll}
				Mikropóry & $<$2 nm \\
				Mezopóry & 2--50 nm \\
				Makropóry & $>$50 nm \\
			\end{tabular}
		\end{center}
	\end{column}
\end{columns}	
	\vfill
}

\subsection{Rtuťová porozimetrie}
\frame{
	\frametitle{}
	\vfill
	\begin{columns}
		\begin{column}{.6\textwidth}
			\textbf{Rtuťová porozimetrie}
			\begin{itemize}
				\item Do vzorku je vtláčena rtuť.
				\item Jde o destruktivní metodu.
				\item Umožňuje měřit velikost pórů od 4~nm do stovek mikrometrů.
				\item Čím vyšší tlak působí, tím se dostává rtuť do menších pórů, spodní hranici ovlivňuje maximální možný tlak.
				\item Měření je poměrně rychlé.
				\item Problémem je toxicita rtuti.
			\end{itemize}
		\end{column}
		\begin{column}{.5\textwidth}
			\begin{figure}
				\adjincludegraphics[width=.78\textwidth]{img/Hg-porosimeter.png}
				\caption*{Rtuťový porozimetr.\footnote[frame]{Zdroj: \href{https://www.quantachrome.com/pdf_brochures/07128.pdf}{Quantachrome}}}
			\end{figure}
		\end{column}
	\end{columns}
	\vfill
}

\frame{
	\frametitle{}
	\vfill
	\adjincludegraphics[width=\textwidth]{img/Hg-porosimetry.png}
	\vfill
}

\subsection{Plynová porozimetrie}
\frame{
	\frametitle{}
	\vfill
	\begin{columns}
		\begin{column}{.7\textwidth}
			\textbf{Plynová porozimetrie}
			\begin{itemize}
				\item Stanovení porozity pomocí sledování sorpce plynu.
				\item Využívá se dusík, argon a krypton.
				\item Založeno na fyzisorpci -- nedochází k chemickým reakcím.
				\item Tlaky se pohybují mezi atmosférickým tlakem a vakuem.
				\item Měření trvá hodiny až dny a probíhá při teplotě varu daného plynu.
			\end{itemize}
			\begin{tabular}{|l|l|}
				\hline
				\textbf{Plyn} & \textbf{Teplota varu [$^\circ$C]} \\\hline
				Dusík & $-$195 \\\hline
				Argon & $-$185 \\\hline
				Krypton & $-$152 \\\hline
			\end{tabular}
		\end{column}
		\begin{column}{.3\textwidth}
			\begin{center}
				\adjincludegraphics[height=.8\textheight]{img/Kyvety.jpg}
			\end{center}
		\end{column}
	\end{columns}
	\vfill
}

\frame{
	\frametitle{}
	\vfill
	\begin{columns}
		\begin{column}{.5\textwidth}
			\begin{itemize}
				\item Před měřením je nutné vzorek (přesně navážený) odplynit (degasovat).
				\item To se provádí zahříváním ve vakuu, doba a teplota závisí na konkrétním vzorku.
				\item Teplota by neměla překročit 80 \% teploty tání nebo skelného přechodu, aby nedocházelo k~povrchovým změnám.
				\item Teplotu je nutné volit i s~ohledem na teplotní stabilitu vzorku.
			\end{itemize}
		\end{column}
		\begin{column}{.5\textwidth}
			\begin{center}
				\adjincludegraphics[height=.8\textheight]{img/Degas2.jpg}
			\end{center}
		\end{column}
	\end{columns}
	\vfill
}

\frame{
	\frametitle{}
	\vfill
	\begin{columns}
		\begin{column}{.5\textwidth}
			\begin{itemize}
				\item Na začátku měření je vzorek evakuován a ochlazen na měřící teplotu.
				\item Přístroj provede automatickou kalibraci -- stanovení \textit{cold volume} a \textit{warm volume}.
				\item Do kyvety se vkládá skleněná tyčinka, která zmenšuje velikost mrtvého objemu.
				\item Pro měření malých pórů je nutné dosáhnout velmi nízkého tlaku, k tomu se využívá \textit{turbomolekulární vývěva}.
			\end{itemize}
		\end{column}
		\begin{column}{.5\textwidth}
			\begin{center}
				\adjincludegraphics[height=.8\textheight]{img/Autosorb-iQ3.jpg}
			\end{center}
		\end{column}
	\end{columns}
	\vfill
}

\frame{
	\frametitle{}
	\vfill
	\begin{columns}
		\begin{column}{.65\textwidth}
			\begin{itemize}
				\item Byla vynalezena roku 1958 Dr. W.Beckerem.\footnote[frame]{\href{https://www.chemeurope.com/en/encyclopedia/Turbomolecular_pump.html}{Turbomolecular pump}}
				\item Skládá se ze soustavy statických a rotujících lopatek.
				\item Rotující lopatky se pohybují velmi vysokou rychlostí (25~000--90 000~rpm).
			\end{itemize}
		\end{column}
		\begin{column}{.4\textwidth}
			\begin{figure}
				\adjincludegraphics[height=.5\textheight]{img/Cut_through_turbomolecular_pump.jpg}
				\caption*{Řez turbomolekulární vývěvou.\footnote[frame]{Zdroj: \href{https://commons.wikimedia.org/wiki/File:Cut_through_turbomolecular_pump.jpg}{Liquidat/Commons}}}
			\end{figure}
		\end{column}
	\end{columns}
	\vfill
}

\frame{
	\frametitle{}
	\vfill
	\begin{columns}
		\begin{column}{.6\textwidth}
			\begin{itemize}
				\item Vyžaduje předřazenou vývěvu pro vytvoření dostatečného vakua pro start. Tlak by měl být pod 10 Pa.
				\item Hřídel s lopatkami je umístěna v magnetickém ložisku.
				\item Umožňují dosažení tlaku až 10$^{-9}$~Pa a čerpací rychlosti až 4~000~l.s$^{-1}$.
			\end{itemize}
		\end{column}
		\begin{column}{.4\textwidth}
			\begin{figure}
				\adjincludegraphics[height=.5\textheight]{img/Turbo_pump_schematic-2011-05-02.png}
				\caption*{Schéma turbomolekulární vývěvy.\footnote[frame]{Zdroj: \href{https://commons.wikimedia.org/wiki/File:Turbo_pump_schematic-2011-05-02.gif}{Kkmurray/Commons}}}
			\end{figure}
		\end{column}
	\end{columns}
	\vfill
}

\frame{
	\frametitle{}
	\vfill
	\begin{columns}
		\begin{column}{.5\textwidth}
			\begin{itemize}
				\item Přístroj postupně zvyšuje tlak plynu v kyvetě a měření objem nasorbovaného plynu, který se projevuje poklesem tlaku.
				\item Aby bylo možné měřit tlak s dostatečnou přesností potřebujeme velice přesný manometr a velmi dobře kalibrovaný manifold.
				\item V manifoldu se nastaví vyšší tlak a po otevření kyvety dojde k poklesu tlaku, který lze dopředu spočítat ze známého objemu kyvety a manifoldu.
			\end{itemize}
		\end{column}
		\begin{column}{.5\textwidth}
			\begin{center}
				\adjincludegraphics[height=.8\textheight]{img/Mereni.png}
			\end{center}
		\end{column}
	\end{columns}
	\vfill
}

\frame{
	\frametitle{}
	\vfill
	\begin{columns}
		\begin{column}{.5\textwidth}
			\begin{itemize}
				\item Každý bod analýzy má zadány hodnoty \textit{equilibration} a \textit{tolerance}.
				\item Equilibration udává, jak dlouho bude přístroj čekat na ustavení rovnováhy.
				\item Tolerance udává rozptyl hodnot tlaků.
				\item Jakmile je bod změřen, zvýší se tlak a měří se další.
				\item Zpravidla měříme adsorpčně-desorpční izotermu.
			\end{itemize}
		\end{column}
		\begin{column}{.5\textwidth}
			\begin{center}
				\adjincludegraphics[width=\textwidth]{img/Equilibration.png}
			\end{center}
		\end{column}
	\end{columns}
	\vfill
}

\frame{
	\frametitle{}
	\vfill
	\begin{itemize}
		\item Příkladem mezoporézního materiálu je hexagonální silikát \textit{MCM-41}.
		\item Jeho struktura sestává z válcovitých pórů o průměru 2 až 6~nm.
		\item Připravit ho lze vodnou sol-gelovou syntézou, jako zdroj křemíku slouží zpravidla TEOS.
		\item Velikost pórů můžeme ovlivnit volbou templátu, často se používá \textit{cetyltrimethylammonium bromid}.
	\end{itemize}
	\begin{figure}
		\adjincludegraphics[width=\textwidth]{img/MCM41-Synthesis.png}
		\caption*{Syntéza mezoporézního materiálu MCM-41.\footnote[frame]{Zdroj: \href{https://commons.wikimedia.org/wiki/File:MCM-41_Synthesis_English_2014.04.19.svg}{Hermann Luyken/Commons}}}
	\end{figure}
	\vfill
}

\frame{
	\frametitle{}
	\vfill
	\begin{figure}
		\adjincludegraphics[width=\textwidth]{img/MCM41-isotherm.png}
		\caption*{Adsorpčně-desorpční izoterma MCM-41}
	\end{figure}
	\vfill
}

\frame{
	\frametitle{}
	\vfill
	\begin{figure}
		\adjincludegraphics[width=.78\textwidth]{img/Adsorption_of_Gases_in_Multimolecular_Layers.jpg}
		\caption*{Adsorpce plynů na povrchu.\footnote[frame]{Zdroj: \href{https://commons.wikimedia.org/wiki/File:Adsorption_of_Gases_in_Multimolecular_Layers.jpg}{Ricardo Amaral/Commons}}}
	\end{figure}
	\vfill
}

\subsection{BET teorie}
\frame{
	\frametitle{}
	\vfill
	\begin{itemize}
		\item Brunauer–Emmett–Teller (BET) -- velmi používaný způsob výpočtu měrného povrchu.
		\item Využívá začátek adsorpční izotermy (P/P$_0$ = 0-0,3), kdy lze předpokládat vznik monovrstvy.
		\item Vychází z několika (bohužel nereálných) předpokladů:
		\begin{itemize}
			\item plochý povrch adsorbentu
			\item všechna adsorpční místa jsou energeticky ekvivalentní (homogenní)
			\item neexistují vzájemné interakce mezi adsorbovanými molekulami
			\item adsorpční energie je pro všechny molekuly vyjma první vrstvy rovna energii zkapalnění
			adsorbátu
			\item neomezený počet adsorpčních vrstev, nekonečný při nasyceném tlaku
			\item rychlost desorpce molekul v určité vrstvě je rovna rychlosti kondenzace ve vrstvě o jednu níže
		\end{itemize}
		\item $\frac{\frac{p}{p_0}}{V(1-\frac{p}{p_0})} = \frac{1}{CV_m} + \frac{C-1}{CV_m}\frac{p}{p_0}$
	\end{itemize}
	\vfill
}

\frame{
	\frametitle{}
	\vfill
	\begin{itemize}
		\item Dříve se pro stanovení BET povrchu využívala adsorpce dusíku při 77 K, dnes se doporučuje argon při 87 K.\footnote[frame]{\href{https://www.3p-instruments.com/wp-content/uploads/2017/04/2015-IUPAC-Technical-Report.pdf}{IUPAC Technical Report}}
		\item Molekula dusíku může zkreslovat výslednou hodnotu kvůli kvadrupolárnímu momentu molekuly dusíku. Může docházet k interakci s polárními místy na povrchu vzorku.
		\item Pro měření s argonem potřebujeme kapalný argon nebo zařízení, které dokáže temperovat kyvetu na teplotu kapalného argonu.
	\end{itemize}
	\begin{columns}
	\begin{column}{.5\textwidth}
		\adjincludegraphics[height=.4\textheight]{img/Cryosync2.png}
	\end{column}
	\begin{column}{.5\textwidth}
		\adjincludegraphics[height=.4\textheight]{img/Cryosync1.png}
	\end{column}
\end{columns}
	\vfill
}

\frame{
	\frametitle{}
	\vfill
	\begin{figure}
		\adjincludegraphics[width=.9\textwidth]{img/BET.png}
		\caption*{Jedenáctibodová izoterma}
	\end{figure}
	\vfill
}

\frame{
	\frametitle{}
	\vfill
	\begin{figure}
		\adjincludegraphics[width=.9\textwidth]{img/BET2.png}
		\caption*{Jedenáctibodová BET křivka}
	\end{figure}
	\vfill
}

\end{document}